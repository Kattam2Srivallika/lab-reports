\section{Results}
In this experiment, we have measured the relative permittivity, or the dielectric constant of different mediums by using two capacitor plates and varying the voltage or distance parameters. The values calculated are as follows:\\

\begin{itemize}
    \item By varying the charge on the capacitor plates, keeping their distance constant, $\varepsilon_\text{air}= (1.032\pm0.061)$
    \item By varying the distance between the plates, keeping the voltage constant, $\varepsilon_\text{air} = (1.181\pm0.082)$
    \item Average dielectric constant of air, $\varepsilon_\text{air} = (1.106\pm0.051)$
    \item Dielectric constant of styrofoam, $\varepsilon_\text{styrofoam} = (2.184\pm0.051)$
    \item Dielectric constant of wood, $\varepsilon_\text{wood} = (3.588\pm0.011)$
\end{itemize}

\section{Conclusion}
The literature value of the dielectric constant of air is 1 and that of styrofoam is 1.03. The dielectric constant of dry wood can range from 1.4 to 2.9. While the value of $\varepsilon_\text{air}$ is quite close to 1, there is significant deviation in the other values.This deviation could be due to various reasons, majorly the moisture. Water is a polar molecule and has a high dielectric constant, which may affect our mesurement. This is especially true of wood, which can possess high moisture content. Other than that, variations in temperature, improper calibration of the measuring instruments or errors in the setupor charge leakage due to imperfect insulation of the capacitor plates could also contribute to the error.

If the dielectric constant is known, one can also of find the \textit{electric susceptibility} of a material ($\chi$), which is directly related to the polarizability of the molecules in the material. Hence, this helps us know more abour the internal structure of any solid.

\section{Precautions}
\begin{enumerate}
    \item The capacitor plates must be dried at regular intervals using a blow dryer, to prevent moisture from collecting.
    \item The high voltage supply must be handled carefully and should be turned off once not in use.
    \item Use short cables as much as possible. Avoid loose connections.
    \item The capacitor plates should not be touched when charged, to prevent any electric shock.
\end{enumerate}