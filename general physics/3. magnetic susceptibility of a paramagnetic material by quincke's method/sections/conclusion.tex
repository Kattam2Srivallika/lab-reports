\section{Conclusion}
The literature value of molar susceptibility of Fe is $1.69\times10^{-8}$ m$^3$/mol. Hence there is a -62.7\% deviation of the actual value from the expected value. This could be due to various reasons, such as error in calibration or instrumental error.

Here $\chi$ comes out to be positive, since FeCl$_3$ is a paramagnetic material. In this case, the magnetic field in the material is strengthened by the induced magnetization. Alternatively, if $\chi$ is negative, the material is diamagnetic. In this case, the magnetic field in the material is weakened by the induced magnetization.

\section{Precautions and Sources of Error}
\begin{enumerate}
    \item Increase the current slowly and carefully. Make sure to limit the current at 4A, or the instrument might heat up.
    \item To avoid backlash error, try to move the travelling microscope in only one direction
    \item The circuit should be connected properly and must be verified before switching on
\end{enumerate}